This thesis presented the results of a search with the CMS detector for new heavy resonances decaying into a pair of vector bosons (WW/WZ) or into a W and a Higgs boson (WH).
Such a search represents a key aspect of the research program of the ATLAS and CMS experiments at the LHC aimed at finding confirmation of the existence of new physics beyond the standard model.
In fact, despite its predictions being experimentally verified with great precision, it is broadly believed that the standard model is an incomplete theory and attempts have been made to propose theoretical solutions able to explain its deficiencies. New theoretical extensions could for instance explain the ``unnaturally'' large difference between the electroweak and the gravitational scales, commonly referred to as hierarchy problem. A prominent feature of these new models is the prediction of new resonances with masses in the TeV range, which can be produced in pp collisions at the LHC thanks to the high energies of the proton beams.
Moreover, these new particles can be directly measured by the LHC experiments by reconstructing their preferential decay into a pair of well-known SM particles, such as a vector and a Higgs boson.

The lepton+jet decay modes of the two SM bosons are exploited in this work, taking advantage of the large rejection of the prominent multijet background achievable thanks to the striking signature of the lepton, together with the high branching fractions provided by the $\PW/\PZ\to\qqbarpr$ and $\PH\to\bbbar$ decays. In addition, these final states allow the invariant mass of the diboson system to be fully reconstructed, such that the spectrum is measured to search for the signal appearing as a local enhancement over a smoothly falling background distribution. On the other hand, these final states are also notably challenging since a resonance with mass of order of a TeV would produce bosons of such large momenta that the particles emerging from the decay would be very collimated. In particular, the hadronization products from the decay of the highly-boosted bosons are contained within a single reconstructed jet such that the bosons must be identified by studying the substructure of this merged jet. Newly-developed and dedicated V-tagging and H-tagging techniques that exploit the substructure of such objects are applied to resolve the decays of the bosons and suppress SM backgrounds, making diboson signatures standard candles in the quest for new physics at the LHC.

The data collected in pp collisions at $\sqrt{s} = 8\TeV$ and corresponding to 19.7\fbinv of integrated luminosity are analyzed within this work to search for a WH resonance in the $\ell\nu\bbbar$ decay channel. The analysis, published in Ref.~\cite{Khachatryan:2016yji}, reported an interesting excess of events corresponding to 2.2 standard deviations with respect to the SM expectations. At about the same time the ATLAS collaboration reported an excess in the all-jets search, corresponding to a local significance of 3.4 standard deviations for a resonance with a mass of 2\TeV. These exciting results, in conjunction with theoretical motivation, have made resonant searches in the diboson final states a flagship of the CMS and ATLAS collaborations. In fact, because of the large interest in the possibility of confirming the presence of a signal, these searches were prioritized in order to provide first results at the restart of pp collisions at $\sqrt{s} = 13\TeV$ in 2015. The analysis of the first 13\TeV data, corresponding to 2.3\fbinv of integrated luminosity, aimed at searching for a WZ or WW resonance in the $\ell\nu\qqbar$ decay channel, has been presented in this thesis. In particular, keeping in mind the interest in confirming a potential signal, the analysis was optimized to provide significant discrimination between signals due to a spin-1 or a spin-2 resonance, as well as a charged or neutral one.
The results, published in Ref.~\cite{CMS-PAS-EXO-15-002}, did not show evidence for a signal and in particular, the observed excesses at a resonance mass of about 2\TeV were not confirmed.
A statistical combination of all the 8 and 13\TeV CMS searches for heavy resonances decaying to a pair of SM bosons has afterwards been performed as presented in this thesis.
For the first time, the overall experimental status of searches with boosted W, Z, and Higgs bosons is compiled, and a large gain in sensitivity is achieved through their combination.
%The large interest in the possibility of confirming the presence of a signal, the searches in these final states have been prioritized 
%This excitement has been further enhanced by the deviation reported in the same mass range by the ATLAS collaboration in a search for heavy diboson resonances in the all-hadronic final state. A large number of 
%Because of the interest raised by these results, the searches in these final states have been

As no evidence for a signal is found in any of these searches, upper limits are set on the production cross section of the resonance under the assumption of a natural width negligible compared to the experimental resolution.
The limits are interpreted in the context of theoretical extensions of the standard model that attempt to solve the aforementioned hierarchy problem.
It has been shown in this thesis that narrow-width, spin-1 resonances with masses up to 2.4\TeV, as predicted by a simplified theoretical approach representing scenarios such as composite Higgs models, have been excluded by these searches. For a narrow-width graviton as predicted by warped extra dimension models, the analyses have not reached yet enough sensitivity to exclude such a signal because of the small values of the predicted production cross sections.

Despite the absence of a signal so far, it is possible that a new resonance will emerge with the large amount of data that will be collected by the experiments in the upcoming collection of LHC data.
In fact, about 40\fbinv of integrated luminosity have been delivered in 2016, and additional data equivalent to approximately 100\fbinv are expected to be collected by the end of 2018.
The large amount of data will allow a more precise measurement of the mass spectra, especially in the high-mass tail where a possible signal is expected, hence improving the sensitivity of this type of searches.
In addition to this, it is of utmost importance in the search for new and rare physical phenomena, that the CMS detector maintains its excellent performance.
This is achieved by constantly monitoring and optimizing its sub-detectors. In particular, important contributions to the pixel barrel sub-detector have been made in the context of this thesis.
The extensive work carried out to install into CMS, commission and calibrate the detector before the beginning of 13\TeV collisions, guaranteed its successful and stable operation during data-taking in 2015 and 2016.
The original pixel detector that took the data analyzed in this thesis has been recently replaced by a new, upgraded system designed to maintain and improve the performance with the upcoming high luminosities of the LHC in the next years.

The installation of the new detector into CMS and its commissioning are the last steps of several years of work carried out by many institutes around the world. In particular, the University of Zurich has been responsible for the design, construction, integration and testing of the service cylinders, a complex system that carries the services along the beam pipe, accommodates the cooling lines and houses the power converters and distribution, and electronics for detector readout and control.
Prior to the assembly of the service cylinders a major effort was put forth to set up a test stand at the University of Zurich with multiple goals. The first goal was to test the prototypes of the detector components prior to full production. Secondly, it has been fundamental in establishing testing and calibration procedures that had been used to verify the functionality of millions of pixel channels during the operations of integration, installation and commissioning of the full system.

The LHC collisions will restart in the summer of 2017, and the upgraded pixel detector will start its operation and continue to 2023, so that the CMS experiment will be able to cope with increased instantaneous and integrated luminosity, and even provide improvements in tracking performance.