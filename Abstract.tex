\vspace*{\fill}

\begin{center}
\Large
\textbf{Abstract}
\end{center}

\noindent This doctoral thesis presents a search for new massive particles decaying to pairs of W, Z, and Higgs bosons performed with the CMS detector at the Large Hadron Collider (LHC).
Such processes are the prominent feature of several extensions of the standard model that aim to clarify open questions in the SM, such as the apparently large difference between the electroweak and the gravitational scales. 
The semi-leptonic final states are considered, in which one of the bosons decays leptonically and the other hadronically. 
The first study is focused on a $\PW\PH$ resonance decaying into the $\ell\nu\bbbar$ final state and based on data recorded in proton-proton (pp) collisions at a center-of-mass energy $\sqrt{s} = 8\TeV$ during 2012 (LHC Run~1).
The second study is focused on a $\PW\PW$ or $\PW\PZ$ resonance decaying into the $\ell\nu\qqbar$ final state and based on 2015 data corresponding to pp collisions at $\sqrt{s} = 13\TeV$ (LHC Run~2).
These final states are particularly challenging because for large resonance masses the bosons are highly energetic and the two quarks from the decay are separated by a small angle in space, resulting in the presence of one single merged jet after hadronization. This jet is identified as coming from a Higgs, W or Z boson applying novel jet substructure techniques and dedicated algorithms for the identification of b jets.
The results for these two studies are finally combined with limits derived in companion CMS searches for resonances decaying to a pair of bosons in several different final states, with data collected in both LHC Run~1 and Run~2.
This is the first combined search for high mass resonances with both WW/WZ and WH/ZH signatures.\\

\noindent Excellent detector performance is of utmost importance to search for new and rare physical phenomena.
The efficient reconstruction of secondary vertices and precise measurements of the track impact parameter rely on this detector.
Hence, the inner tracking pixel detector is a key component to tag events involving Higgs bosons decaying into \bbbar.
Stable performance and future upgrades are thus necessary to maintain high identification efficiency of b jets for the entire lifespan of the LHC.
Various aspects of my contributions to the CMS pixel barrel detector are detailed in the second part of the thesis.
In particular, a major effort has been put after LHC Run~1 to replace and test faulty channels,
and to perform calibrations aimed at optimizing the detector after it has been heavily irradiated during Run~1.
The detector has been re-installed into CMS in December 2014 and the large effort in commissioning and calibration 
resulted in the successful and stable operation of the CMS pixel detector during data-taking in 2015 and 2016.
Despite the excellent performance up to now, the pixel detector has not been designed to cope with the upcoming high luminosities of the LHC in the next years.
Hence, a stepwise upgrade is foreseen, which is referred to as the ``Phase I Pixel Upgrade'', which will be installed in Spring 2017.
A test stand at the University of Zurich has been setup, which includes a slice of the CMS pixel data-acquisition system and all components of the upgraded read-out chain, together with a number of detector modules.
The test system has been fundamental to develop new tests and procedures to be used during the upgraded detector assembly, commissioning and calibration.

\vspace*{\fill}
