\chapter{Conclusions}
\label{ch:summary2}

The contributions to the calibration and upgrade of the CMS pixel barrel detector have been presented in this part of thesis.
First, a major effort was made during the long shut-down to recover the full detector performance after the first LHC physics run.
This effort included the replacement of broken channels and the re-calibration of detector parameters at low temperature needed to compensate for the effects of radiation damage.
The detector was re-installed into CMS in December 2014 and commissioned in January 2015 for the second LHC physics run at the highest center-of-mass energy of 13\TeV.
The entire operation was completed in only a few days thanks to the expertise gained during the long shut-down.
The first pp collision at 13\TeV occurred on March, 21st 2015 and since then the detector
%showed excellent performance.
was running stable with a high data-taking efficiency.
%Its conditions and performance were monitored during the first two years of data-taking after re-installation and they have been excellent.
%This has been made possible thanks to re-calibrations that happened throughout 2015 and 2016.

A second contribution has been made during this thesis to the Phase 1 upgrade pixel project.
The upgrade detector will allow to maintain the excellent tracking performance of CMS at the upcoming higher luminosity conditions at the LHC.
A test stand at the University of Zurich has been set up and run with the aim of testing the performance of the complete upgrade pixel system and gain experience in its operations.
The setup includes a slice of the full CMS pixel DAQ together with all the upgrade electronics for the power, readout and control systems as well as newly developed pixel modules.
The test system has been employed to implement and test new developments in the pixel online software used to operate with the detector.
Although the software architecture remains unchanged, several calibration procedures for the present detector become obsolete with the novel digital readout of the upgrade system,
whereas brand-new tests have been developed. 
Additional fundamental modifications had to be understood and implemented to be able to operate the software with the upgrade detector.
The Phase 1 pixel upgrade project is now at its last stage of assembly and testing of the entire system.
The novel calibration procedures are aimed at guaranteeing the success of these operations, as well as of the installation and commissioning of the new system planned for March 2017.
