\chapter{Diboson resonances as signature for new physics}
\label{ch:dibosonIntro}

This part of the thesis is dedicated to the description and discussion of searches for new physics in proton-proton collision data collected with the CMS experiment at LHC.
As pointed out in Chapter~\ref{ch:theory}, the remarkable compatibility of the discovered scalar resonance by the ATLAS and CMS collaborations with the SM predictions for the Higgs boson,
force physicists to deeply understand the role of naturalness in the dynamics of this particle.
Several theoretical extensions to the SM have been proposed offering a concrete realization of naturalness,
where new particles with masses in the TeV range generate loop corrections with the necessary cancellations to stabilize the Higgs boson mass.
This means that the attention can be restricted to direct experimental manifestations of new physics which consist of the production of reasonably narrow new particles.
More natural solutions can therefore be probed at the LHC through the direct discovery of these new particles in final states with SM objects with known properties.
The research described in this work follows exactly this approach and it is focused on the direct search for new massive resonances ($M_X > 800\GeV$) decaying to
pairs of vector bosons (WW, WZ, or ZZ)
or to a vector boson and a Higgs boson (WH or ZH).
These decay modes are preferred for resonances predicted in several BSM models. 
Popular examples of such models include the bulk scenario of the Randall--Sundrum warped extra-dimensions described in Section~\ref{subsec:graviton},
as well as the composite Higgs and Littlest Higgs models discussed in Section~\ref{subsec:composite}.
Furthermore, the HVT model (Section~\ref{subsec:hvt}) generalizes a large class of explicit models that predict new heavy spin-1 vector bosons,
adopting a simplified model strategy.
The properties of the described benchmark models studied in this thesis are summarized in Table~\ref{tab:models}.

\begin{table}[!htb]
\centering
\caption{Summary of the properties of the heavy resonance models considered in this work. The polarization of the produced W/Z boson in all considered models is mostly longitudinal.}
\resizebox{\textwidth}{!}{
\begin{tabular}{cccccc}
\hline
model & particles & spin & charge & main production & main decay \\
\hline
HVT model A, $g_{V} = 1$ & \Wpr singlet & 1 & $\pm1$ & $\qqbarpr$ & $\qqbarpr$ \\ 
HVT model A, $g_{V} = 1$ & \Zpr singlet & 1 & 0 & $\qqbar$ &  $\qqbar$\\
HVT model A, $g_{V} = 1$ & \Wpr+\Zpr triplet & 1 & 0,$\pm1$ & $\qqbar$/$\qqbarpr$ &  $\qqbar$/$\qqbarpr$\\
\hline
HVT model B, $g_{V} = 3$ & \Wpr singlet & 1 & $\pm1$ & $\qqbarpr$ & WZ,WH \\ 
HVT model B, $g_{V} = 3$ & \Zpr singlet & 1 & 0& $\qqbar$ &  WW,ZH \\
HVT model B, $g_{V} = 3$ & \Wpr+\Zpr triplet & 1 & 0,$\pm1$ & $\qqbar$/$\qqbarpr$ &  WW,WZ,WZ,ZH\\
\hline
%2 / neutral / $G_{\text{RS1}}$ & subdominantly WW, ZZ & mainly $\qqbar$ and $gg$ \\
RS bulk scenario, $\tilde{k}=0.5$ & $\rm G_{bulk}$ & 2 & 0 & $gg$ & WW, ZZ \\
\hline
\end{tabular}}
\label{tab:models}
\end{table}

The signal under investigation is a narrow resonance, referring to the assumption that the resonance?s natural width
is smaller than the experimental resolution, covering a large fraction of the parameter space of the reference models considered. \FIXME{why we do that?}
The search considered here is conducted in semi-leptonic final states, where one of the two bosons is a W boson
decaying into a charged lepton ($\ell$) and a neutrino ($\Pgn$).
The lepton can be either a muon ($\mu$) or an electron (e), however, the results include the W $\rightarrow\tau\Pgn$ contribution from the decay $\tau\rightarrow\Pgn\Pgn$.
The gain in sensitivity from $\tau$ leptons is limited by the small branching fractions involved.
The second boson in the final state decays into hadrons, and can be either a vector boson V = W or Z,
or a Higgs boson. In the first case, the final state is labelled as $\ell\Pgn\qqbar$ including W $\rightarrow\qqbarpr$ and Z $\rightarrow\qqbar$ decays (Figures~\ref{fig:FDsignals_b},~\ref{fig:FDsignals_c} and~\ref{fig:FDsignals_d}).
For the Higgs boson, the final state is labeled as $\ell\Pgn\bbbar$ referring to the Higgs boson decay into a bottom quark-antiquark pair (Fig.~\ref{fig:FDsignals_a}).

\begin{figure}[!htb]
\centering
\subfigure[]{\label{fig:FDsignals_a}\includegraphics[width=0.3\textwidth]{\chfour/WprimeWH_LO.pdf}}
\subfigure[]{\label{fig:FDsignals_b}\includegraphics[width=0.3\textwidth]{\chfour/WprimeWZ_LO.pdf}}\\
\subfigure[]{\label{fig:FDsignals_c}\includegraphics[width=0.3\textwidth]{\chfour/ZprimeWW_LO.pdf}}
\subfigure[]{\label{fig:FDsignals_d}\includegraphics[width=0.3\textwidth]{\chfour/BulkGWW_LO.pdf}}
\caption{Feynman diagrams for the production of a spin-1 charged resonance \Wpr decaying to the $\ell\Pgn\bbbar$ (a) and $\ell\Pgn\qqbar$ (b) final states.
The latter final state is also given by the Feynman diagrams for the production of a spin-1 \Zpr and spin-2 $G$ neutral resonances.}
\label{fig:FDsignals}
\end{figure}

The search in the $\ell\Pgn\bbbar$ final state is based pp collision data at $\sqrt{s} = 8\TeV$ collected in 2012 and corresponding to an integrated luminosity of 19.7\fbinv.
The second analysis described in this thesis and focused on the $\ell\Pgn\qqbar$ final state is instead based on the pp collision data at $\sqrt{s} = 13\TeV$ collected in 2015 and corresponding to an integrated luminosity of 2.3\fbinv.
The analysis strategy is similar in the two searches and specific differences will be emphasized throughout the text.

The key challenge of these analyses is the reconstruction of the highly energetic decay products.
Since the resonances under study have masses of $\approx$ TeV, their decay products, i.e. the bosons,
have on average transverse momenta of several hundred GeV or more.
As a consequence, the particles emerging from the boson decays are very collimated.
In particular, the jet-decay products of the bosons cannot be resolved using the standard algorithms,
but are instead reconstructed as a single jet object. Dedicated techniques, so-called jet ``V tagging'' and ``H tagging'' techniques,
are applied to exploit the substructure of such jet objects, and can help resolve jet decays of massive bosons.
These techniques also help to suppress SM background, which mainly originates from the production of W bosons in association with jets (W+jets).

The aim is to reconstruct the full event to be able to search for a localized enhancement in the invariant mass of the WV or WH system
on the top of a smoothly falling SM background distribution.
The background mainly comprises W+jets production, while another significant contribution is represented by 
events involving pair produced top quarks (\ttbar).
Other minor backgrounds are represented by single top quark and SM diboson (WW, WZ or ZZ) production processes.

The invariant mass of the WV and WH system is determined by estimating the neutrino transverse momentum with the measured missing transverse energy
(\ETmiss) in the event, while an estimate of the neutrino longitudinal momentum is derived by imposing the constraint of the W mass on the invariant mass of the $\ell\Pgn$ system.
In the following the diboson invariant mass will be labelled either $m_{\ell\Pgn\mathrm{j}}$ system, or $m_{WV}$ and $m_{WH}$ for the $\ell\Pgn\qqbar$ and $\ell\Pgn\bbbar$ final states, respectively.
The mass spectrum for the dominant W+jets background is determined from observed events with a reconstructed jet mass not compatible with the V or H hypothesis. 
This strategy partially relies on the simulation of the background processes.
Furthermore, simulated events are used for the optimization of the analysis selection aimed at maximizing the discrimination of the signal against the background and hence the analysis sensitivity.

This part of the thesis is organized as follows.
Chapter~\ref{ch:dataAndSim} gives an overview of the methods used to simulated the physics processes happening in pp collisions at the LHC
together with a description of the specific simulated background and signal events used in these analyses, as well as a discussion about the data sets analyzed.
Chapter~\ref{ch:EventReconstruction} provides a detailed description of the algorithms used in CMS for the reconstruction of the event and of the physics objects expected in the semi-leptonic final states under investigation.
Particular attention is given to the V and H tagging algorithms representing the key feature of these analyses and separately discussed in Chapter~\ref{ch:vtagging}.
The analysis strategy, already outlined here, is explained in details in Chapter~\ref{ch:strategy}.
This includes the final event selection and categorization optimized to enhance the analysis sensitivity,
as well as the strategy for the estimation of the expected background, the modelling of the signal and the related systematic uncertainties
which will be used as input to the statistical analysis of the diboson invariant mass distribution observed in data.
The final results are discussed in Chapters~\ref{ch:results8} and~\ref{ch:results13} for the 8 and 13\TeV data analysis, respectively. 

Eventually, the results of these analyses are combined with limits derived in companion CMS searches for resonances decaying to a pair of bosons in several different final states during both LHC Run 1 and Run 1.
These analyses use the same V and H tagging techniques as presented here to separate the signal from the large multijet or V+jets background.
The statistical combination represent the last piece of this work and it is presented in Chapter~\ref{ch:combination}.