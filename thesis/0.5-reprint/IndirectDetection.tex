\subsection{Indirect Detection}
\label{secIndirectDetection}

Particle dark matter can be detected indirectly, by observing radiation produced in its annihilation reactions: $\gamma$-ray, synchrotron radiation, neutrino, positron and anti-proton fluxes~\cite{BertoneHooper}. Since the flux is proportional to annihilation rate, which depends on the square of the dark matter density, it is commonly searched in the regions where dark matter densities are expected to accumulate through inelastic collisions, such as the galactic center in the Sun or in dwarf galaxies.

High energy $\gamma$-rays, in the GeV$-$TeV energy range relevant for dark matter searches, can be directly detected only by space telescopes, since $\gamma$-rays will interact with matter via production of electron-positron pairs, with an interaction length much shorter than the thickness of the atmosphere~\cite{IndirectDetection_gamma}. The Energetic Gamma-Ray Experiment Telescope (EGRET) was one of four detectors on the Compton Gamma Ray Observatory (CGRO), which measured a diffused $\gamma$-emission from the galactic plane and discovered an excess above $\sim$1~GeV~\cite{Indirectdetection_EGRETexcess}. From the spectral shape, a WIMP mass of 50$-$100~GeV has been estimated~\cite{IndirectDetection_EGRETinterpretation}, which, however, has been ruled out by measurements of the anti-proton flux, and the observation might  be due to an inaccurate estimation of the detector sensitivity~\cite{IndirectDetection_EGRETexplanation}. The next generation Fermi Large Area Telescope (LAT, former GLAST) has been launched into orbit and is currently taking data~\cite{IndirectDetection_Fermi}.

On the ground, $\gamma$-rays can be indirectly detected with air shower detectors, observing the night sky for Cerenkov light emitted from charged particles produced in high energy $\gamma$-interactions in the upper atmosphere. These detectors, called Imaging Atmospheric Cerenkov Telescopes (IACTs), are sensitive in the energy range from $\sim$100~GeV to several TeV, and examine the $\sim$10~km high region, where the showers reach their maximum intensity. Examples of experiments are HESS~\cite{IndirectDetection_HESS} and MAGIC~\cite{IndirectDetection_MAGIC}, while the next-generation ground-based $\gamma$-ray observatory, the Cerenkov Telescope Array (CTA), is already being designed~\cite{IndirectDetection_CTA}.

WIMPs can annihilate in the core of the Sun or the Earth into muon neutrinos, which can escape and produce ultra-relativistic muons in charged current interactions with nuclei in terrestrial targets. They can be detected in large water- or ice-based neutrino telescopes. The most stringent bounds on high energy neutrinos from the Sun and the Earth come from the Super-Kamiokande~\cite{IndirectDetection_SuperK} experiment, AMANDA~\cite{IndirectDetection_AMANDA}, and its successor IceCube~\cite{IndirectDetection_IceCube}.

A cosmic anti-matter experiment, the satellite-borne charged cosmic ray detector PAMELA, has reported an excess of anti-protons~\cite{IndirectDetection_PAMELA1}, which may indicate annihilation of dark matter in the galactic halo, or may be caused by the production of positrons by near-by pulsars~\cite{IndirectDetection_PAMELA2}. 
