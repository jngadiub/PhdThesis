\chapter{Introduction}
\label{chIntroduction}

Several astronomical observations of the 20th century, such as redshift measurements of the galaxy clusters, gravitational lensing effects, and studies of the rotational velocity of spiral galaxies, have revealed anomalies at different scales, that can only be explained by a deviation from the theory of gravity, or by assuming the existence of a large amount of non-luminous, `invisible' matter. In our present understanding of the Universe, only $\sim$5\% of the matter is ordinary (luminous and baryonic). The rest of the matter is in a form of `dark energy' (72\%) and `dark matter' ($\sim$23\%). Chapter~\ref{chParticleDarkMatter} is dedicated to particle dark matter, which provides a window into physics beyond the standard model. 
The observational evidences that the Universe is not dominated by ordinary matter are presented in Section~\ref{secObservationalEvidence}, and the most popular dark matter candidates are introduced in Section~\ref{secTheoreticalPredictions}. Among them are Weakly Interacting Massive Particles (WIMPs), which appear naturally in theories with supersymmetry and universal extra dimensions. The search for particle dark matter is ongoing  in collider experiments, and by direct and indirect detection techniques, which are presented in Section~\ref{secParticleDMsearches}.

WIMPs may be directly detected by scattering off heavy nuclei in a very sensitive detector. The XENON100 detector, which is the main subject of this thesis, is one of them, being the second generation detector within the XENON dark matter search program. 
It is fully operational and taking science data since end of 2009 underground in the Laboratori Nazionali del Gran Sasso (LNGS), Italy. The detection principle, the design of the detector, its shield, the cryogenic and gas purification, as well as data acquisition systems, are described in Chapter~\ref{chXe100detector}. A detailed detector model, developed within the GEANT4 framework for various Monte Carlo simulations, is presented in Section~\ref{secGeant4model}.

XENON100 employs photomultiplier tubes (PMTs) to detect scintillation light produced by particle interaction in the liquid xenon target. Chapter~\ref{chLightDetection} describes their arrangement within the detector, and explains room temperature tests to determine their functionality and the calibration system developed for regular measurements of the relevant PMT characteristics during detector operation.

The XENON100 detector is a time-projection chamber (TPC) that allows three-dimensional reconstruction of the event vertex. Chapter~\ref{chPositionReconstruction} describes an algorithm developed for this purpose based on a neural network, and presents the results of the analysis to validate the performance of the algorithm on measured and simulated data.

The calibration of the detector, performed with external and internal radioactive sources is introduced in Chapter~\ref{chDetectorCalibration}. In particular, a combined energy scale providing good linearity and resolution for $\gamma$-spectroscopy is discussed. One of the addressed points is the measurement of the veto efficiency measurements and its implementation into the GEANT4 Monte Carlo model.

The predicted signal rates are below one interaction per kg of target material and day, hence one of the crucial requirements for the detector is an extremely low background level. Chapter~\ref{chERbackground} is dedicated to measurements of the radioactive contamination in the materials used for detector construction with germanium spectrometers and mass-spectrometry, and to measurements of the intrinsic radioactivity in the liquid xenon using delayed time coincidence techniques. Furthermore, it gives detailed predictions of the electromagnetic backgrounds from all known sources, and presents a spectroscopy analysis of the measured background spectrum. 
Monte Carlo studies of the radiogenic and cosmogenic neutron backgrounds have been also performed, and are described in Chapter~\ref{chNRbackground}.

Finally, details and results of the first two dark matter searches with XENON100 are presented in Chapter~\ref{chDarkMatterSearch}. Here, the nuclear recoil energy scale is explained, together with the calibration of the electronic and nuclear recoil distributions, which allow for background discrimination.

The concluding remarks are given in Chapter~\ref{chConclusions}, summarizing the performed tasks and data analysis. The next step of the XENON dark matter search program is introduced here, and the expected sensitivity and background goal are discussed.

%The first result on WIMP-nucleon [7] shows no WIMP candidates seen in the pre-defined signal region. The 90\% confidence upper limit excludes all parameter space for the interpretation of the CoGeNT and DAMA signals as being due to low mass WIMPs, and demonstrates the potential of the XENON100 detector to discover WIMP Dark Matter.
