\vspace*{\fill}

\begin{center}
\Large
\textbf{Zusammenfassung}
\end{center}

\small
\noindent Die vorliegende Doktorarbeit stellt eine Suche nach neuartigen schweren Teilchen am CMS Detektor am Large Hadron Collider (LHC) vor. Die Suche befasst sich mit dem Zerfall dieser neuartigen Teilchen in Paare von W, Z oder Higgs Bosonen. Diese Zerf\"alle sind ein wichtiges Merkmal verschiedener Erweiterungen des Standard Modells, die beabsichtigen offene Fragen des SM, wie zum Beispiel die deutliche Differenz zwischen der elektroschwachen Skala und der Planck-Skala, zu beantworten. Die Suche basiert auf semileptonischen Zerf\"allen, in denen eines der Bosonen leptonisch und das andere hadronisch zerf\"allt. Die erste Analyse befasst sich mit $\PW\PH$ Resonanzen im Endzustand $\ell\nu\bbbar$ und benutzt die Daten, die im Jahr 2012 bei Proton-Proton Kollisionen mit einer Schwerpunktsenergie von $\sqrt{s} = 8\TeV$ (LHC Run~1) aufgezeichnet wurden. Die zweite Analyse befasst sich mit $\PW\PW$ und $\PW\PZ$ Resonanzen im Endzustand $\ell\nu\qqbar$ und benutzt die Daten, die im Jahr 2015 bei Proton-Proton Kollisionen mit einer Schwerpunktsenergie von $\sqrt{s} = 13\TeV$ (LHC Run~2) aufgezeichnet wurden. Die Rekonstruktion dieser Zerf\"alle ist \"aussert anspruchsvoll. Aufgrund der grossen Masse der Resonanz haben die Bosonen eine sehr hohe Energie, wodurch die Quarks im hadronischen Zerfall in einen kleinen Raumwinkel emittiert werden und sich zu einem einzigen Teilchenjet im Detektor zusammenf\"ugen. Um diesen Teilchenjet als das Zerfallsprodukt von Higgs, W oder Z Bosonen zu identifizieren, werden neuartige Methoden angewandt, um die Unterstruktur der Teilchenjets aufzul\"osen und spezielle Algorithmen benutzt, um b-Quarks zu identifizieren. Die Resultate der beiden Analysen werden mit den Resultaten weiterer CMS Analysen, die in anderen Zerfallskan\"alen nach massiven Resonanzen suchen, kombiniert. Die Kombination beinhaltet die Daten von LHC Run~1 und Run~2.  Die hier vorgestellte Arbeit ist die erste Suche nach massiven Resonanzen, die sowohl WW/WZ als auch WH/ZH Signaturen behandelt.\\

\noindent Der einwandfreie Betrieb des CMS Detektors ist unerl\"asslich f\"ur die Suche nach neuen und seltenen physikalischen Ph\"anomenen. Insbesondere beruht sie auf der effizienten Rekonstruktion von Sekund\"arvertices und der pr\"azisen Messung des Stossparameters von rekonstruierten Spuren. Der Pixeldetektor im Innersten des CMS Detektors ist folglich eine Schl\"usselkomponente, um Ereignisse mit Higgs Bosonen, die in Paare von b-Quarks zerfallen, zu identifizieren. Um die hohe Rekonstruktionseffizienz von b-Quarks am CMS Detektor w\"ahrend der gesamten Laufzeit des LHCs zu gew\"ahrleisten, sind eine zuverl\"assige Datennahme und sp\"ater neue und verbesserte Detektoren erforderlich. Meine vielf\"altigen Beitr\"age zum CMS Barrel Pixeldetektor werden im zweiten Teil dieser Arbeit besprochen. Dabei ist insbesondere das Testen und die Reparatur des Detektors nach LHC Run~1 zu erw\"ahnen, sowie die optimale neue Kalibration, die den Effekt allf\"alliger Strahlensch\"aden mildert.
Der Pixeldetektor wurde im Dezember 2014 wieder in CMS installiert und die Anstrengungen, die unternommen wurden, um den Detektor in Betrieb zu nehmen und zu kalibrieren, bilden die Grundlage f\"ur die zuverl\"assige und \"ausserst erfolgreiche Datennahme w\"ahrend 2015 und 2016. Trotz des bis anhin einwandfreien Betriebs des Pixeldetektors, ist er nicht daf\"ur geschaffen, die bevorstehenden hohen Luminosit\"aten der n\"achsten Jahre am LHC zu bew\"altigen. Deshalb sind schrittweise Verbesserungen des Detektors vorgesehen. Der erste verbesserte Detektor, der sogenannte "Phase 1 Pixel Upgrade", wird im Fr\"uhling 2017 installiert. An der Universit\"at Z\"urich wurde ein Testsystem f\"ur den Phase 1 Pixeldetektor aufgebaut, das einen Teil des CMS Datennahmesystems, sowie alle Komponenten der Ausleseelektronik des Pixeldetektors und einige Detektormodule umfasst. Die wichtigsten neuen Tests und Prozeduren, die f\"ur den Bau, die Inbetriebnahme und die Kalibration des Phase 1 Pixeldetektors ben\"otigt werden, sind an diesem Testsystem entwickelt worden. 

\vspace*{\fill}
