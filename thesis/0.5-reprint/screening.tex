\section{Radioactive Contamination in the Detector and Shield Components}
\label{secScreening}

Special care has been taken to select detector and shield materials according to their low radioactive contamination. Before detector construction, the majority of materials planned to be used were screened with low background Ge detectors in order to determine their intrinsic radioactivity, mostly due to residual $^{238}$U, $^{232}$Th, $^{40}$K, and $^{60}$Co contaminations.

The XENON experiment has access to a dedicated screening facility underground at LNGS, the Gator detector~\cite{gator}, operated by the UZH group. This is a high purity coaxial germanium detector with a sensitive mass of 2.2~kg, energy resolution of $\sim$3~keV FWHM at 1332~keV, and large sample cavity with 19 liters volume. Moreover, the LNGS screening facility, including some of the most sensitive Ge detectors in the world \cite{LNGSfacility}, such as GeMPI-I and GeMPI-II~\cite{GeMPI}, and the inductively coupled plasma-mass spectrometry method (ICP-MS), have also been used. More than 20 different materials, and in some cases several batches of a given material, have been examined.

The radioactive contamination of these materials is shown in Table~\ref{tabScreeningResults}. Details on the XENON100 screening campaign have been published in Ref.~\cite{ScreeningPaper}.

\begin{table}[!t]
\centering
\caption[Radioactive contamination in the materials used to construct the XENON100 detector and shield ]{Radioactive contamination in the components of the XENON100 detector and its shield from measurements at underground facilities at LNGS~\cite{ScreeningPaper}, as used for Monte Carlo simulations. The upper limits from the measurements have been treated as fixed values.}
\label{tabScreeningResults}
%\vspace{0.2cm}
%\begin{tabular}{l|l|c|c|c|c|l}
\begin{tabular}{>{\footnotesize}l|>{\footnotesize}c|>{\footnotesize}c|>{\footnotesize}c|>{\footnotesize}c|>{\footnotesize}l}
\hline
Component 					& \multicolumn{5}{>{\footnotesize}c}{Total radioactive contamination [mBq]} \\
							& $^{238}$U / $^{226}$Ra	& $^{232}$Th 	 	& $^{60}$Co 	 	& $^{40}$K 	 	& other nuclides \\
\hline
Cryostat and `diving bell' (316Ti SS) 	& 121.46			& 147.23			& 404.87			& 662.52			&\\
Support bars (316Ti SS) 			& 64.58 			& 144.07			& 69.55 			& 352.73 			&\\
Detector PTFE 					& 0.71 			& 1.19			& 0.36			& 8.89 			&\\
Detector copper 				& 0.85 			& 0.62 			& 5.21 			& 0.78 			&\\
PMTs						& 60.50			& 111.32			& 181.50 			& 1972.30 		& $^{137}$Cs: 41.14 \\
PMT bases 					& 38.72			& 16.94			& 2.42			& 38.72 			&\\
TPC resistor chain 				& 1.11 			& 0.57			& 0.12			& 7.79 			&\\
Bottom electrodes (316Ti SS) 		& 0.43			& 0.45 			& 2.14			& 2.36 			&\\
Top electrodes (316Ti SS) 		& 0.85			& 0.43 			& 1.73 			& 1.16 			&\\
PMT cables					& 0.85 			& 1.97			& 0.37 			& 18.65 			& $^{108m}$Ag: 2.67 \\
\hline
Copper shield 					& 170.80 			& 24.69			& 6.59 			& 80.26			&\\ 
Polyethylene shield 				& 368.0 			& 150.4 			& - 				& 1120.0 		&\\
Lead shield  (inner layer) 			& 4.3$\times$10$^{3}$ 		& 3.6$\times$10$^{3}$			& 7.2$\times$10$^{2}$ 			& 9.6$\times$10$^{3}$ 		& $^{210}$Pb: 1.7$\times$10$^{8}$ \\
Lead shield (outer layer) 			& 1.1$\times$10$^{5}$		& 1.4$\times$10$^{4}$			& 2.9$\times$10$^{3}$			& 3.8$\times$10$^{5}$			& $^{210}$Pb: 1.4$\times$10$^{10}$ \\
\hline
\end{tabular}
\end{table}

There is a possibility to verify secular equilibrium within the $^{232}$Th decay chain, due to the relatively high branching ratios of the $\gamma$-lines emitted in the initial part of the chain (i.e. $^{228}$Ac). For the $^{238}$U decay chain, the branching ratios for $\gamma$-emitters ($^{234}$Th, $^{234m}$Pa) are very low, and for most of the screened samples the upper limits on the activity of $^{238}$U are much higher than for $^{226}$Ra, which are obtained by measuring prominent $\gamma$-lines from Pb and Bi decays.  However, the hypothesis of secular equilibrium cannot be excluded, and the results in the tables are presented under this assumption.

The precise measurement of $^{235}$U is not possible, because its 185.72~keV highest intensity $\gamma$-line overlaps with the line from $^{226}$Ra decay at 186.10~keV. Thus, the activity of $^{235}$U for predictions of nuclear recoil background (Section~\ref{chNRbackground}) is deduced from that of $^{238}$U, by taking into account their relative abundance in natural uranium (0.70\% of $^{235}$U and 99.27\% of $^{238}$U). 

The radioactive contamination in the cryostat and the diving bell  has been calculated taking into account the screening results for the batches of 316Ti SS of the different thickness, presented in Table~\ref{tabStainlessSteelRA}, and the mass of the individual components from Table~\ref{tabStainlessSteel}. If spectral lines have not been observed, the result of the screening is presented as an upper limit at 95\% confidence level.
A contamination with $^{54}$Mn (T$_{1/2}$ = 312.23~days), produced by cosmic ray activation during surface exposure, has been detected in all the steel samples, with activities ranging from (0.5$\pm$0.2)~mBq/kg to (1.7$\pm$0.4)~mBq/kg sample (see Table~\ref{tabStainlessSteelRA}). The Monte Carlo study of this background source and comparison with the measurements are discussed in Section~\ref{secCosmogenicActivationSteel}.

\begin{table}[!b]
\centering
\caption[Radioactive contamination measured for the batches of 316Ti stainless steel with different thickness]{Radioactive contamination measured for the batches of 316Ti stainless steel with different thickness~\cite{ScreeningPaper}. The isotopes $^{54}$Mn, $^{58}$Co, and $^{48}$V are produced in steel by cosmogenic activation.}
\label{tabStainlessSteelRA}
%\vspace{0.2cm}
\begin{tabular}{>{\footnotesize}c|>{\footnotesize}c|>{\footnotesize}c|>{\footnotesize}c|>{\footnotesize}c|>{\footnotesize}c|>{\footnotesize}c|>{\footnotesize}c}
%\begin{tabular}{c | c | c | c | c | c}
\hline
Thickness [mm] 	& \multicolumn{7}{>{\footnotesize}c}{Radioactive contamination [mBq/kg]} \\
				& $^{238}$U / $^{226}$Ra	& $^{232}$Th 	 	& $^{60}$Co 	 	& $^{40}$K 	& $^{54}$Mn		& $^{58}$Co		& $^{48}$V\\
\hline
1.5 				& $<$ 1.9					& $<$ 1.0  		& 8.5$\pm$0.9		& 10$\pm$4  	& 0.7$\pm$0.4		& $<$0.62			& 0.5$\pm$0.2 \\
2.5 				& $<$ 2.7					& $<$ 1.5			& 13$\pm$1		& $<$ 12  		& 0.5$\pm$0.2		& 0.5$\pm$0.2		& 0.2$\pm$0.1 \\
3.0 				& 3.6$\pm$0.8				& $<$ 1.8			& 7$\pm$1		& $<$ 5.7 		& 1.36$\pm$0.24 	& 0.44$\pm$0.02	& NA \\
25 				& $<$ 1.3					& 2.9$\pm$0.7		& 1.4$\pm$0.3		& $<$ 7.1 		& 1.7$\pm$0.4		& $<$0.67			& $<$0.57 \\
\hline
\end{tabular}
\end{table}

\begin{table}[!b]
\centering
\caption[Radioactive contamination in the PMMA-PFA fibers]{Radioactive contamination in the PMMA-PFA fibers used for PMT calibration with an LED light, measured with an inductively coupled plasma-mass spectrometry method~\cite{ScreeningPaper}.}
\label{tabScreeningFibers}
%\vspace{0.2cm}
\begin{tabular}{>{\footnotesize}c|>{\footnotesize}c|>{\footnotesize}c}
%\begin{tabular}{ c | c | c}
\hline
$^{238}$U			& $^{232}$Th 	 		& $^{40}$K \\
\hline
6$\pm$2~mBq/kg		& $<$ 1.9~mBq/kg 		& 40$\pm$8~mBq/kg  	\\
\hline
\end{tabular}
\end{table}

The radioactive contamination of the PMMA-PFA optical fibers, installed in the detector for PMT calibration, has been measured with the ICP-MS method. The results are presented in Table~\ref{tabScreeningFibers}. Taking into account their low mass ($\sim$2~g), the contribution to the total background is negligible.


The results of the measurements have been used for Monte Carlo simulations of the electronic recoil background, and of the nuclear recoil background due to neutron production in ($\alpha$,n) and spontaneous fission reactions.


%Secular equilibrium can be verified for 232Th because of the relatively high branching ratios of the ?-lines emitted in the ini- tial part of the chain. For most of the screened samples this is not possible for 238U. However, upper limits obtained from the ?-lines	of	the	daughters	234m Pa	and	234 Th	do	not	exclude	the secular equilibrium hypothesis. 

