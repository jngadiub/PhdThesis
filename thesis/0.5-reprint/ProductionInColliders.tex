\subsection{Production in Colliders}
\label{secProductionInColliders}

High energy particle accelerators, such as the Large Hadron Collider (LHC) or the Tevatron, might be able to produce WIMPs and to find dark matter~\cite{Colliders_1, Colliders_3}. The main advantages of collider searches are sensitivity to low dark matter masses, and that they do not suffer from astrophysical uncertainties.

Phenomenological consequence of R-parity is that supersymmetric particles can only be produced in colliders in pairs in the collision of ordinary matter. 
Since WIMPs are stable on the order of the lifetime of the Universe, they do not decay within the detector volume. 
Hence, dark matter candidates appear as `missing energy'~\cite{Colliders_1}. 

A typical collision involves quarks and gluons carrying only a small fraction of the parent energy, which implies that cross sections fall dramatically with the mass of produced states. Hence, light states can be produced with large rates, and constraints fall off for WIMP masses exceeding the typical energy reach of the colliders. The collider sensitivity to spin-dependent interactions is stronger than that of direct searches over a significant portion of parameters space, especially for light WIMP masses, where sensitivity of direct detection experiments is limited by energy thresholds~\cite{Colliders_3}.  However, while collider searches can constrain WIMP mass quite well, they do not have high sensitivity for its scattering cross section. Hence, dark matter searches at high energy particle colliders must be complemented by direct and indirect detection experiments. 
In addition, even if a WIMP is discovered in a collider, the existence of the galactic dark matter halo will still have to be proven.


% Indeed, for many operators, the direct detection rates are expected to be very small because of the velocity suppression, and colliders become the only way to effectively probe WIMP-hadron interactions. In the case of a WIMP whose dominant recoil is through a spin-dependent interaction, collider constraints are already much stronger even than the expected reaches of near-future direct detection experiments. Thus, if such an experiment were to observe a positive signal, the collider constraints would immediately imply a break- down of the effective field theory at collider energies, revealing the existence of a light mediator particle.

