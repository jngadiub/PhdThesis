\chapter{Conclusions}
\label{ch:summary1}

A search for new massive resonances decaying into a pair of vector bosons (WW/WZ) or into a W boson and a Higgs boson (WH)
in lepton+jet final states has been presented. In particular two analyses and a statistical combination with previous searches have been described.

The first analysis is performed with pp collision data at $\sqrt{s} = 8\TeV$ collected in 2012, and is focused on the final state given by the W boson decay to $\ell\nu$, with $\ell$ = $\mu$ or e,
and the Higgs boson decay to a pair of bottom quarks.
The second analysis is performed with pp collision data at $\sqrt{s} = 13\TeV$ collected in 2015, and also in this case a final state is considered given by the $\PW\to\ell\nu$ decay together with the decay of the second boson into quarks, where the second boson (V) can be either a W or a Z. 

In both analyses, each event is reconstructed as a leptonic W-boson candidate recoiling against a jet with mass compatible with the H- or V-boson mass for the $\ell\nu\bbbar$ or $\ell\nu\qqbar$ analysis channel, respectively.
Specialized methods, referred to as V tagging and H tagging, are exploited to help resolve jet decays of massive bosons and achieve large suppression of background from the W+jets process.
In particular, the H-tagging algorithm combines jet-substructure information with identification techniques based on the peculiarities of b jets.

In the $\ell\nu\bbbar$ analysis channel, no excess of events above the standard model prediction is observed in the muon channel,
while an excess with a local significance of 2.9 standard deviations is observed in the electron channel at $\mWH \approx 1.8\TeV$.
Taking into account the look-elsewhere effect, the results are statistically compatible with the standard model within 2 standard deviations.
In the context of the little Higgs and the heavy vector triplet models, upper limits at 95\% CL are set on the \Wpr production cross section
in a range from 100 to 10\unit{fb} for masses between 0.8 and 2.5\TeV, respectively.
Within the little Higgs model, a lower limit on the \Wpr mass of 1.4\TeV has been set.
A heavy vector triplet model that mimics the properties of composite Higgs models has been excluded up to a \Wpr mass of 1.5\TeV.

These results are improved by the limits set by the analysis in the $\ell\nu\qqbar$ decay channel.
No evidence for a signal is found in this search with new 2015 data, and the result is interpreted as an upper limit on the production cross section of a narrow-width resonance as a function its mass,
in the context of several benchmark models for spin-1 and spin-2 resonances. In particular, for the same heavy vector triplet model as mentioned above the data exclude a $\Wpr$ resonance with masses $< 1.9\TeV$.

However, the best results are provided by a statistical combination of all searches performed in CMS with 8 and 13\TeV data for massive resonances decaying to pairs of W, Z, and Higgs bosons in various final states.
The results are interpreted in the context of heavy vector singlet and triplet models predicting a \Wpr and a \Zpr decaying to WZ, WH, WW, and ZH and a model with a bulk graviton that decays into WW and ZZ.
The combined significance of a potential resonances at 1.8--2.0\TeV has been evaluated and has been found to be 0.8 standard deviations for the hypothesis of a \Wpr,
thus the excesses observed in the $\ell\nu\bbbar$ channel in 8\TeV data is not supported.
The combination yields mass limits at the 95\% CL on spin-1 resonance in the range 2.2--2.4\TeV, depending on the specific benchmark.
The most stringent cross section limits on a narrow-width bulk graviton resonance with $\tilde{k} = 0.5$ to date are set in the mass range from 0.6 to 4\TeV.
