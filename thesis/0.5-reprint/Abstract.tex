\chapter*{Abstract}
\label{chAbstract}

The XENON100 experiment searches for the direct evidence of dark matter in the form of Weakly Interacting Massive Particles (WIMPs). It is installed and taking data underground at Laboratory Nazionali del Gran Sasso (LNGS) in Italy. The existence and nature of dark matter is one of the most important questions of modern physics, which provides a window into physics beyond the Standard Model. The present thesis described many achievements, related to the development and construction of the experiment, Monte Carlo simulations,  background studies and data analysis.

Chapter~\ref{chLightDetection} describes the room temperature tests of hundreds of photomultiplier tubes (PMTs), of which 242 were selected to be installed in the detector. These tests have been performed in order to determine the PMTs functionality and arrangement within the detector according to the measured characteristics. The hardware and software systems developed for regular calibrations during the detector operation and for the monitoring of their response are introduced.

The precise modeling of the scintillation light collection has been performed with Monte Carlo simulations, and have been verified by comparison with the measured data. This allowed development of an algorithm for the event vertex reconstruction with millimeter precision. It is based on a pattern recognition with a neural network, and is described in Chapter~\ref{chPositionReconstruction}.

The level of electromagnetic background of $<$10$^{-2}$~events$\cdot$kg$^{-1}\cdot$day$^{-1}\cdot$keV$^{-1}$, achieved in the XENON100 experiment, is much lower than in any of the existing direct dark matter detection experiments, which makes it a very sensitive device for WIMP detection. The studies of electromagnetic background have been performed using spectroscopic and a delayed coincidence analyses of the measured data, and by Monte Carlo simulations, and are presented in Chapter~\ref{chERbackground}. The background sources are identified and the background spectrum is explained in the full energy range accessible to the XENON100 detector.

The predictions of the nuclear recoil background from radiogenic and cosmogenic neutrons, performed with Monte Carlo simulations, are described in Chapter~\ref{chNRbackground}. The resulting background level is $<$0.5~events/year, which does not limit the sensitivity goal of the experiment.

The first two dark matter search analyses, presented in Chapter~\ref{chDarkMatterSearch}, result in the limits on spin-independent WIMP-nucleon scattering cross sections (with the minimum at 7.0$\times$10$^{-45}$ cm$^{2}$ at a WIMP mass of 50~GeV/c$^{2}$), which are the best achieved by direct dark matter detection experiments at the time of writing.
